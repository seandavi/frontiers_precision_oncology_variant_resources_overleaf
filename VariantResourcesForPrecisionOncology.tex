\documentclass{article}

\usepackage[letterpaper, margin=1in]{geometry}
\usepackage{tabularx}
\usepackage{fancyhdr}
\usepackage[textsize=tiny]{todonotes}
\usepackage{hyperref}
\usepackage[backend=biber]{biblatex}
\usepackage{authblk}

\title{Resources for Interpreting Variants in Precision Genomic
Oncology Applications}

\addbibresource{references.bib}


\author[2,3]{Hsinyi Tsang}
\author[2,3]{Durga Addepalli}
\author[1]{Sean R. Davis\thanks{Address correspondence to: seandavi@gmail.com}}
\affil[1]{Center for Cancer Research, 
  National Cancer Institute, 
  National Institutes of Health, 
  Bethesda, MD, USA}
\affil[2]{Center for Biomedical Informatics and Information Technology, 
  National Cancer Institute, 
  National Institutes of Health, 
  Gaithersburg, MD, USA}
\affil[3]{Attain, LLC, 
  McClean, VA, USA} 

\renewcommand\Authands{ and }

\begin{document}

\maketitle


\begin{abstract} Precision genomic oncology--applying high throughput
sequencing (HTS) at the point-of-care to inform clinical decisions--is
a developing precision medicine paradigm that is seeing increasing
adoption. Simultaneously, new developments in targeted agents and
immunotherapy, when informed by rich genomic characterization, offer
potential benefit to a growing subset of patients. Multiple previous
studies have commented on methods for identifying both germline and
somatic variants. However, interpreting individual variants remains a
significant challenge, relying in large part on the integration of
observed variants with biological knowledge.  A number of data and
software resources have been developed to assist in interpreting
observed variants, determining their potential clinical actionability,
and augmenting them with ancillary information that can inform
clinical decisions and even generate new hypotheses for exploration in
the laboratory. Here, we review available variant catalogs, variant
and functional annotation software and tools, and databases of
clinically actionable variants that can be used in an ad hoc approach
with research samples or incorporated into a data platform for
interpreting and formally reporting clinical results.
\end{abstract}


\section{Introduction}
Genomic technologies and approaches have transformed
cancer research and have led to the production of large-scale cancer
genomics compendia
\cite{noauthor_undated-vx,Cancer_Genome_Atlas_Research_Network2013-gt}. The
resulting molecular characterization and categorization of individual
samples from such compendia has driven development of molecular
subtypes cancers as well as enhanced understanding of the molecular
etiologies of carcinogenesis
\cite{Cancer_Genome_Atlas_Network2012-nz,Cancer_Genome_Atlas_Research_Network2015-gd,noauthor_2008-wg}. The
development of novel and effective targeted therapies has proceeded in
parallel with and been accelerated by deeper, faster, and broader
genomic characterization \cite{Blumenthal2016-fb}, enabling early
application of molecular characterization at the point of care to
inform clinical decision-making
\cite{Flaherty2012-dq,Shaw2013-wl,Maemondo2010-dj,Druker2006-qk} and
to address resistance to primary therapy \cite{Ai2014-nf}. Genomic
characterization also has applications in immune approaches to
cancer. Chimeric antigen receptor T-cell (CARt) therapy have shown
great success in diseases with well-characterized antigens that are
relatively tumor-specific \cite{Grupp2013-nd}. Previously referred to
as precision oncology \cite{Sohal2015-bi}, genomics-driven oncology
\cite{Garraway2013-zo}, genomic oncology, and even simply as precision
medicine, the paradigm of applying high-throughput genomic approaches
to patient samples is rapidly changing the landscape of oncology care
and clinical oncology research.

Conventional approaches to clinical trials design may be inadequate
due to molecular heterogeneity of tumors derived from a single primary
tissue \cite{Simon2016-ik}, leading to the adoption of basket,
umbrella, and hybrid trials designs. A number of studies are ongoing
to determine feasibility and potential impact of precision genomic
oncology at the point-of-care
\cite{Cheng2015-wh,noauthor_undated-da,Lopez-Chavez2015-cg}. In
addition to studies focused on identifying targetable mutations,
immune-based therapeutic approaches are also being informed by HTS
applied to patient samples
\cite{Bethune2017-ns,Chalmers2017-ya,Faltas2016-yz}.

As with any clinical testing modality, whether in a research setting
or at the point-of-care, a clear understanding of the goals of
applying the test is necessary when first designing the test and its
validation. However, the flexibility and number of potential data
items that arise from even a limited application of HTS has lead the
US Food and Drug Administration (FDA) has begun to define its
regulatory role \cite{Fda2015-kv} and, critically, how existing
knowledge bases can be applied in real time to address findings from
clinical HTS testing \cite{Fda2016-kx}.

This review aims to provide
an organized set of biological knowledge bases with relevance to the
interpretation of small variants, defined as single nucleotide
variants or short (on the order of 20 base pairs or fewer) insertions
and deletions. The catalogs of observed variants section lists
large-scale catalogs of variants, useful for filtering known common
polymorphisms and identifying previously-identified cancer
variants. When a variant observed in a clinical sample has not been
seen but appears to affect the protein coding sequence, the functional
annotation resources section presents a sampling of some of the most
common software and databases for predicting the impact on protein
function. Several data products have been developed to provide
decision support (with strong disclaimers and caveats) directly
linking observed variants to clinical intervention in point-of-care
HTS applications. Integrating the various data sources described in
this review with variants observed in individual patients can be
accomplished with combinations of software tools for the manipulation
of variant datasets.

\subsection{Catalogs of observed germline and somatic variants}

Databases of observed variation in normal populations,
diseased individuals, and cancer compendia form the map onto which
observed variants in patients are projected. Because of the vast
quantities of genomic data and, specifically, DNA variants, there is a
tension between providing rich, highly-curated information about
individual variants and producing the largest possible catalog of
variants with manageable levels of curation. This section reviews some
of the available catalogs (Table \ref{table:1}) of genomic variation observed
in the germline as well as those that appear in tumors as somatic
mutations.  Note that many of the databases mentioned below overlap in
data sources (some nearly completely), but they may differ in the
amount and depth of curation, additional metadata added to each
variant, speed of updates, and methods or formats for access.

\begin{table}[p]
\centering
\begin{tabularx}{\textwidth}{XXXp{1.7cm}}
  \hline
Resource & Variant Type & URL & Citation \\
  \hline
  dbSNP & Germline and somatic & \url{https://www.ncbi.nlm.nih.gov/projects/SNP/} & \cite{Sherry2001-li} \\
  gnomAD  & Germline & \url{http://gnomad.broadinstitute.org/} & \cite{Lek2016-bb} \\
  69 genomes from CGI & Germline & \url{http://www.completegenomics.com/public-data/69-genomes/} & \cite{Drmanac2010-od} \\
  Personalized Genome Project & Germline & \url{http://www.personalgenomes.org/} & \cite{Church2005-lr} \\
  ClinVar & Germline predisposition and somatic & \url{https://www.ncbi.nlm.nih.gov/clinvar/intro/} & \cite{Landrum2016-ul} \\
  NCI Genomic Data Commons & Germline and somatic & \url{https://portal.gdc.cancer.gov/} & \cite{Grossman2016-sk} \\
  cBioPortal & Somatic & \url{http://www.cbioportal.org} & \cite{Cerami2012-el,Gao2013-li} \\
  Intogen (Partial TCGA dataset) & Somatic & \url{https://www.intogen.org/search} & \cite{Rubio-Perez2015-ek,Gonzalez-Perez2013-cl} \\
  Pediatric Cancer Genome Project  & Somatic & \url{http://explorepcgp.org} & \cite{Downing2012-do} \\
   \hline
\end{tabularx}
\caption{catalogs of germline and somatic variants}
\label{table:1}
\end{table}

\subsection{Germline}

Comprehensive catalogs of germline variants inform decisions
about the frequency of variants as seen in the general population as
well as to identify variants that are annotated as
cancer-associated. In the context of tumor sequencing, common variants
are unlikely to be genomic drivers of carcinogenesis and are often
filtered from a report of potential somatic variants. This filtering
process is particularly important when tumor sequencing is not
accompanied by matched normal sequencing. Additional germline
databases that catalog disease-associated variants can be useful to
begin to address familial risk and potentially pharmacogenomic loci
\cite{Wheeler2013-dn,Relling2015-ie}.

Perhaps the oldest of the variant catalogs, dbSNP contains 325,658,303
individual variant records (build 150, accessed May 30, 2017) and is
available in multiple formats, searchable, and linked to records in
literature and other data resources and databases. While the vast
majority of variants in dbSNP have been observed in individuals
without cancer, somatic variants are included and annotated in the
database. Because dbSNP is driven by community submission of variants,
levels of evidence vary among individual variants. The genome
Aggregation Database, or gnomAD, \cite{Lek2016-bb,noauthor_undated-of}
contains information from 123,136 exomes and 15,496 whole-genomes from
unrelated individuals sequenced as part of various disease-specific
and population genetic studies (accessed May 30, 2017). These data
were collected by numerous collaborations, underwent standard
processing, and unified quality control and results area accessible as
a searchable online database and as a downloadable VCF-format text
file. ClinVar \cite{Landrum2016-ul}, maintained by the NIH National
Center for Biotechnology Information (NCBI), is a freely available
archive for interpretations of clinical significance of variants for
reported conditions. Entries in ClinVar are taken directly from
submitters and represent the relationship between variants and
clinical significance. When multiple submissions concerning a single
variant are available, ClinVar supplies high-level summaries of
agreement or disagreement across submitters. Importantly, though,
clinical significance in ClinVar is reported as supplied by the
submitter. The Personalized Genome Project \cite{Church2005-lr}
provides a limited number of fully open-access genome sequencing
results provided by volunteers with trait surveys and even some
microbiome surveys of participants. A catalog of germline variants
derived from 69 genomes sequenced using the Complete Genomics
sequencing platform \cite{Drmanac2010-od} may be useful for groups who
have data generated from the same platform, particularly for
identifying sequencing-platform-specific false positive results.

\subsection{Somatic}

Whereas databases of germline variants are useful to filter
out variants unlikely to be directly involved in carcinogenesis,
databases of somatic variants are useful to identify variants and
their frequencies as observed in tumors. In some cases, identified
variants may be associated with specific tumor types, offering
mechanistic clues, particularly in the rare cancer setting where
biological understanding may be limited.

Several catalogs of somatic variants have, at their core, variants
derived from The Cancer Genome Atlas (TCGA). These databases vary in
the pipelines used to define the variants, the level of annotation
associated with individual variants, the proportion of TCGA included,
and methods for accessing or querying. Recently, National Cancer
Institute (NCI) has established the Genomic Data Commons (GDC) to
harmonize clinical information and genomic results across enterprise
cancer datasets \cite{Grossman2016-sk}, particularly those funded by
NCI, such as TCGA. In addition to the adult tumors profiled as part of
the TCGA, the NCI GDC also contains data from several pediatric tumors
profiled as part of the Therapeutically Applicable Research To
Generate Effective Treatments (TARGET) project
\cite{noauthor_undated-ax}. Cancer cell line data from the Cancer Cell
Line Encyclopedia (CCLE) are also included \cite{Barretina2012-yz} in
the GDC data collection. The GDC is a modern data platform that
provides multiple access methods including a programmatic application
programming interface (API), data file download, and web browser based
text and graphical queries and visualization. The International Cancer
Genome Consortium (ICGC) is a large, international collaboration with
a collection of 76 studies (including TCGA studies) encompassing 21
tissue primary sites. Like the NCI GDC, the ICGC data portal provides
modern data platform approaches to data access, visualization, and
query \cite{Zhang2011-bl}. The Catalog of Somatic Mutations in Cancer
(COSMIC) database is perhaps the largest and best-known cancer variant
database. It presents a unified dataset consisting of curated cancer
variants for specific genes as well as genomic screens from projects
such as TCGA. Several other cancer variant data resources are listed
in Table 1.

\section{Functional Annotation Resources}

When faced with variants with little or no literature or database
support, differentiating those that are deleterious, perhaps
contributing to carcinogenesis, versus those that likely are tolerated
by the cell is a critical task, particularly in the setting of
clinical precision genomic oncology.  A number of algorithms and
methods have been developed to predict the effect of these variants on
protein structure and function as well as the potential for clinical
impact. These prediction methods utilize features of the variant and
its context such as sequence identity, sequence conservation,
evolutionary relationship, protein primary and secondary structure,
entropy based protein stability and approaches such as clustering
based on sequence alignments and machine learning. Some of them are
specific to the type of variant or mutation, some to a disease type,
and some more general. Therefore, applying these functional
annotational tools and interpreting the results in a clinical or
research setting may require significant human curation before being
recognized as clinically actionable. Here we present a review of a
representative set of approaches for predicting pathogenicity of
different variants. For a comprehensive list of prediction tools and
their details see Table \ref{table:2}. For more detailed scientific and technical
explanations of these methods, we refer the reader to a comprehensive
review \cite{Addepalli2014-oa}.

\begin{table}[p]
\centering
\begin{tabularx}{\textwidth}{p{3cm}Xp{1.7cm}X}
  \hline
Resource & URL & Citation & Notes \\
  \hline
 &  &  &  \\
  PolyPhen-2 & \url{http://genetics.bwh.harvard.edu/pph2} & \cite{Adzhubei2013-nj} & Bayesian classification \\
  SIFT & \url{http://sift.jcvi.org} & \cite{Ng2003-vp} & Alignment scores \\
  MutationAssessor & \url{http://mutationassessor.org} & \cite{Reva2011-en} & evolutionary conservation,  naive Bayes classifier \\
  MutationTaster & \url{http://www.mutationtaster.org} & \cite{Schwarz2014-ep} &   \\
  PROVEAN & \url{http://provean.jcvi.org/index.php} & \cite{Choi2012-tk} &   \\
  CADD & \url{http://cadd.gs.washington.edu} & \cite{Kircher2014-im} &   \\
  GERP++ & \url{http://mendel.stanford.edu/SidowLab/downloads/gerp/index.html} & \cite{Davydov2010-ui} &   \\
  PhyloP and PhastCons & \url{http://compgen.cshl.edu/phast/index.php} & \cite{Siepel2005-ke,Pollard2010-of} &   \\
  nsSNPAnalyzer & \url{http://snpanalyzer.uthsc.edu/} & \cite{Bao2005-jn} & Random Forest \\
  SNPs\&GO & \url{http://snps-and-go.biocomp.unibo.it/snps-and-go/} & \cite{Calabrese2009-mi} & SVM \\
  SNAP2 & \url{https://rostlab.org/services/snap2web/} & \cite{Hecht2015-ti} & Neural Networks \\
  SNPs3D & \url{http://www.snps3d.org/} & \cite{Yue2006-kj} & Structure and sequence analysis \\
  MutPred2 & \url{http://mutpred.mutdb.org/} & \cite{Pejaver2017-lr} & Random Forest \\
  AUTO-MUTE & \url{http://binf2.gmu.edu/automute/} & \cite{Masso2010-gf} & Topology  and statistical contact potential, and machine-learning techniques \\ 
  Panther & \url{http://www.pantherdb.org/tools/csnpScoreForm.jsp} & \cite{Thomas2003-rj} & Hidden Markov Model \\
  stSNP & \url{http://ilyinlab.org/StSNP/} & \cite{Uzun2007-bk} & comparative modelling of protein structure \\
  Integrated Predictive Methods &  &  &   \\
  Condel & \url{http://bg.upf.edu/fannsdb/} & \cite{Gonzalez-Perez2011-gc} & a weighted average of the normalized scores from multiple methods \\
  CoVEC & \url{https://sourceforge.net/projects/covec/files} &   &   \\
  CAROL & \url{http://www.sanger.ac.uk/science/tools/carol} & \cite{Lopes2012-je} & combines information from PolyPhen-2 and SIFT \\
  Cancer Specific Prediction tools &  &  &   \\
  CHASM & \url{http://wiki.chasmsoftware.org/index.php/Main_Page} & \cite{Carter2009-ci} & Random Forest, cancer mutations from COSMIC and other cancer-related resources \\
  CanDrA & \url{http://bioinformatics.mdanderson.org/main/CanDrA\#CanDrA} & \cite{Mao2013-ie} & 96  structural, evolutionary and gene features  \\
   \hline
\end{tabularx}
\caption{Tools, software, and databases for functional prediction and annotation of variant impact. }
\label{table:2}
\end{table}

\subsection{PolyPhen-2}

PolyPhen-2, Polymorphism
Phenotyping v2, predicts impact of genomic variants (coding
nonsynonymous SNPs) on protein structure and function and annotates
them \cite{Adzhubei2013-nj}. It is a based on the actual version of
the tool called PolyPhen and is available as batch query web service,
as well as a standalone software. The new features in this version
include high quality multiple sequence alignment pipeline,
probabilistic classifier based on machine-learning method and
optimization for high-throughput analysis of the NGS data. It uses
eight sequence-based and three structure-based predictive features,
which were selected automatically by an iterative greedy
algorithm. The sequence based features include Position-Specific
Independent Counts (PSIC) scores and Multiple Sequence Alignment (MSA)
properties, and position of mutation in relation to domain boundaries
as defined by Pfam \cite{Bateman2004-da}. The structure-derived
features are solvent accessibility, changes in solvent accessibility
for buried residues, and crystallographic B-factor. Majority of these
features involve comparison of a property of the wild-type (ancestral,
normal) allele and the corresponding property of the mutant (derived,
disease-causing) allele, which together define an amino acid
replacement. However, when there are not enough structural parameters,
its classification is based predominantly on comparative
analysis. Thus, structural attributes are complementary to
evolutionary ones, rather than overlapping. PolyPhen-2 predicts the
effect of mutation using a naive Bayesian classifier.

\subsection{SIFT}

SIFT, Sorting Intolerant From Tolerant, predicts functional impacts of
amino acid substitutions (Ng and Henikoff 2003). It is one of the
earliest prediction tools developed and uses sequence homology to
classify amino acid substitutions as tolerated or deleterious and the
prediction is based on conservation built purely on orthologous
protein alignments. SIFT is used as a benchmark for other methods
because of its efficient predictive power, simplicity and ease of
installation and use. SIFT takes into consideration the type of amino
acid change involved and the position at which the change/mutation
occurs. SIFT generates a multiple alignment between the query and
related proteins and uses Dirichlet mixtures extracted from these
protein multiple sequence alignments (PMSAs) to create position
specific scoring matrices (PSSM) and score missense
substitutions. Depending on the amino acids appearing at each position
in the alignment, SIFT calculates the probability for each of the 19
amino acid changes to be tolerated relative to the most frequent amino
acid being tolerated. If this normalized value is less than a cutoff,
the substitution is predicted to be deleterious. However, such a
prediction could be unreliable if there are few homologs
available. Better predictions are obtained if the users can provide
their own curated alignments. SIFT scores of 0.05 are usually taken as
indicative of deleterious substitutions. However, the authors point
out that in some situations higher or lower cutoffs might give a more
accurate result for binary deleterious/neutral classifications.


\subsection{Mutation Assessor}

Mutation Assessor is a cancer-specific fully automated and implemented
computational server/protocol using a functional impact score (FIS)
based on evolutionary information useful for ranking mutations by
likely functional impact. The use of evolutionary information in this
conservation-based approach differentiates it from other
sequence-based predictors.  The method generates a conservation score
by distinguishing between conservation patterns within aligned
families and specificity scores for subfamilies of homologs, therefore
attempts to account for functional shifts between subfamilies of
proteins. The novelty of this approach lies in harnessing the
evolutionary conservation in protein subfamilies, which are determined
by clustering multiple sequence alignments of homologous sequences
with the background of conservation of overall function. Given a
mutated protein name and a mutated residue position, Mutation Assessor
searches for sequence homologs, builds a multiple sequence alignment,
clusters sequences into subfamilies and scores a mutation by global
and subfamily specific conservation patterns. Mutations found in
conserved residues of either groups are likely to be functional. Based
on the assumptions that evolutionarily unfavorable residues are not
observed or observed less frequently than neutral or critically
important residues, while critically important residues are conserved
in diverse evolutionary settings and that the distribution of residues
in any (aligned) sequence position of a protein family can be treated
independently of other positions, the protocol uses the entropy of the
residue distribution in an alignment column as a measure of residue
conservation and estimates the mutation impact (conservation score)
using the difference of the entropy caused by the mutation. To refine
the assessment of conservation patterns, patterns of a subtler type
are considered, in which the evolutionary constraint on a residue type
in a particular position is not constant in the entire family, but
only appears to operate in a protein subfamily. A combinatorial
entropy approach is used to quantify subfamily conservation patterns
which simultaneously determines protein subfamilies, by clustering,
and residues, called specificity residues, which characteristically
differ between these subfamilies. Specificity residues are conserved
within a subfamily but differ between subfamilies presumably encoding
functional diversity. Interestingly, specificity residues were found
to be predominantly located in binding interfaces on the protein
surface implicating them in protein interaction.

\subsection{CHASM}

CHASM, Cancer-specific High-throughput Annotation of Somatic Mutations
(Carter H et al., 2009) is a computational method that identifies and
prioritizes the missense mutations to generate functional changes that
enhance tumor cell proliferation. It is an open-source software, a
collection of Python and C++ programs that takes a list of somatic
missense mutations as input and ranks them according to their likely
tumorigenic impact. The idea behind creating CHASM was to train a
classifier with improved specificity by representing passenger
missense mutations by in silico simulations using mutation profiles
that reflected tumor type as well as mutation context instead of
nsSNPs with high minor allele frequencies. A dataset of driver
mutations comprising of 2,488 missense mutations and identified as
playing a functional role in oncogenic transformation from breast,
colorectal, and pancreatic tumor re-sequencing studies and the COSMIC
database was used. Synthetic passenger mutations were generated in
these genes in silico, using an algorithm that summarized the type of
base substitutions found in brain tumors. The method used only genes
that were mutated as the substrate for the in silico generation of
synthetic mutations such that there were more chances the new
classifier would detect extraordinary mutations over extraordinary
genes. The features and data sets were used to design a new classifier
using two machine learning methods, SVMs, and Random Forests. In a
different work CHASM was compared with PolyPhens PSIC score, SIFT,
CanPredict, KinaseSVM in the fraction of mutations that could be
evaluated, specificity, sensitivity, and precision and proven to be
superior.

\subsection{PROVEAN}

PROVEAN, Protein Variation Effect Analyzer, is a
web server predicting the functional impact of single or multiple
amino acid substitution or insertions and deletions
\cite{Choi2012-tk}. The approach involves clustering of BLAST hits
performed by CD-HIT using 75% global sequence identity. A supporting
sequence set comprising of the top 30 clusters of closely related
sequences is generated and used to generate the prediction. An
alignment score for each supporting sequence is computed and then
averaged within and across clusters to get the final score. If the
score is equal or below a predefined threshold, the protein variant is
predicted to have a "deleterious" effect. If the PROVEAN score is
above the threshold, the variant is predicted to have a "neutral"
effect. The web server currently supports three functions: PROVEAN
Protein, PROVEAN Protein Batch and PROVEAN Genome Variants.

\subsection{CONDEL}

CONDEL, CONsensus DELeteriousness score, is an integrated prediction
method for missense mutations, is a weighted average of the normalized
scores from multiple methods. It integrates the output of two
computational tools, Mutation Assessor and FATHMM, assessing the
impact of non synonymous SNVs on protein function. It computes a
weighted average of the scores (WAS) of these tools. The scores from
different methods are weighted using the complementary cumulative
distributions produced by the five methods on a dataset of
approximately 20000 missense SNPs, both deleterious and neutral. The
probability that a predicted deleterious mutation is not a false
positive of the method and the probability that a predicted neutral
mutation is not a false negative are employed as weights. Condel
scores can be derived for a limited set of specified mutations via the
corresponding web application. The Ensembl database provides
position-specific Condel predictions that combine SIFT and Polyphen-2
for every possible amino acid substitution in all human proteins.

\section{Clinical Actionability}


\begin{table}[p]
\centering
\begin{tabularx}{\textwidth}{p{3cm}XXp{1.5cm}p{1.5cm}}
  \hline
  Cancer Driver Log & \url{https://candl.osu.edu/} & \cite{Damodaran2015-so} & Yes & \\
  Cancer Genome Interpreter & \url{https://www.cancergenomeinterpreter.org/home} & \cite{Tamborero2017-ay} & Yes & API \\
  CIViC & \url{https://civic.genome.wustl.edu/home} & \cite{Griffith2016-sy} & Yes & API \\
  DGIdb & \url{http://dgidb.genome.wustl.edu/} & \cite{Wagner2016-fs,Griffith2013-uv} & Yes & \\
  Clinical Knowledge Base & \url{https://www.jax.org/clinical-genomics/clinical-offerings/ckb\#} & & & \\
  My Cancer Genome & \url{http://www.mycancergenome.org} & \cite{Micheel2014-pz} & Yes & API, app \\
% OncoKb
% http://oncokb.org/
% \cite{Chakravarty2017-gx}
 
% API
% Personalized 
% Cancer Therapy
% https://pct.mdanderson.org
% Not Available
 
 
% PharmGKB
% https://www.pharmgkb.org/
% \cite{Hewett2002-yu}
% Yes
 
% Precision Medicine Knowledge Base
% (Beta)
% https://pmkb.weill.cornell.edu/
% \cite{Huang2016-zx}
% Yes
 
% TumorPortal
% http://www.tumorportal.org/
% \cite{Lawrence2014-ss}
 
% Cancer Driver Log
% https://candl.osu.edu/
% \cite{Damodaran2015-so}
% Yes
 
% Cancer Genome
% Interpreter
% https://www.cancergenomeinterpreter.org/home
% \cite{Tamborero2017-ay}
% Yes
% API
% CIViC
% https://civic.genome.wustl.edu/home
% \cite{Griffith2016-sy}
% Yes
 
% API
% DGIdb
% http://dgidb.genome.wustl.edu/
% \cite{Wagner2016-fs,Griffith2013-uv}
% Yes
 
% Clinical Knowledge Base
% https://www.jax.org/clinical-genomics/clinical-offerings/ckb#
 
 
 
% My Cancer Genome
% http://www.mycancergenome.org
% \cite{Micheel2014-pz}
% Yes
% API, app
% OncoKb
% http://oncokb.org/
% \cite{Chakravarty2017-gx}
 
% API
% Personalized 
% Cancer Therapy
% https://pct.mdanderson.org
% Not Available
 
 
% PharmGKB
% https://www.pharmgkb.org/
% \cite{Hewett2002-yu}
% Yes
 
% Precision Medicine Knowledge Base
% (Beta)
% https://pmkb.weill.cornell.edu/
% \cite{Huang2016-zx}
% Yes
 
% TumorPortal
% http://www.tumorportal.org/
% \cite{Lawrence2014-ss}
 
  
% Cancer Driver Log
% https://candl.osu.edu/
% \cite{Damodaran2015-so}
% Yes
 
% Cancer Genome
% Interpreter
% https://www.cancergenomeinterpreter.org/home
% \cite{Tamborero2017-ay}
% Yes
% API
% CIViC
% https://civic.genome.wustl.edu/home
% \cite{Griffith2016-sy}
% Yes
 
% API
% DGIdb
% http://dgidb.genome.wustl.edu/
% \cite{Wagner2016-fs,Griffith2013-uv}
% Yes
 
% Clinical Knowledge Base
% https://www.jax.org/clinical-genomics/clinical-offerings/ckb#
 
 
 
% My Cancer Genome
% http://www.mycancergenome.org
% \cite{Micheel2014-pz}
% Yes
% API, app
% OncoKb
% http://oncokb.org/
% \cite{Chakravarty2017-gx}
 
% API
% Personalized 
% Cancer Therapy
% https://pct.mdanderson.org
% Not Available
 
 
% PharmGKB
% https://www.pharmgkb.org/
% \cite{Hewett2002-yu}
% Yes
 
% Precision Medicine Knowledge Base
% (Beta)
% https://pmkb.weill.cornell.edu/
% \cite{Huang2016-zx}
% Yes
 
% TumorPortal
% http://www.tumorportal.org/
% \cite{Lawrence2014-ss}
 
\end{tabularx}
\caption{Tools, software, and databases for functional prediction and annotation of variant impact. }
\label{table:3}
\end{table}
 
 


The ultimate goal for many of the above-mentioned resources is to
develop an individualized approach to the diagnosis, prevention and
treatment of cancer, or precision oncology. However, despite recent
advances in HTS, determining the clinical relevance of experimentally
observed cancer variants remains a challenge in the application of HTS
in clinical practice. Difficulties in differentiating driver and
passenger mutations, lack of standards and guidelines in reporting and
interpretation of genomic variants, lack of clinical evidence in
associating genomic variants to clinical outcome, lack of resources to
disseminate clinical knowledge to the cancer community and the precise
definition of actionability have been reported to contribute to the
bottleneck
\cite{Li2017-aw,Prawira2017-gv,Uzilov2016-ct,Hedley_Carr2016-ul}. Comprehensive
resources linking experimentally determined cancer variants and
clinical actionability have been developed to address some of these
challenges and address various aspects of translating research results
into clinical valuable information to support clinical decisions in
precision oncology (See Table 3). In recognition of the fact that
central curation of information regarding actionability is extremely
challenging, several of the resources below use crowdsourcing as a
means of gathering updates and enhancing curation efforts. In addition
to a web interface, some tools provide additional access via API,
mobile app, and/or social media tagging to facilitate dissemination of
information and enhance accessibility. While some of these tools share
similar functions, in the section below, we highlight distinct
features and capabilities for a representative set of resources.

Clinical Interpretation of Variants in Cancer (CIViC) is an open
access and open source platform for community-driven curation and
interpretation of cancer variants. It is based on a crowdsourcing
model where individuals in the community can contribute to produce a
centralized knowledge base with the goal of disseminating knowledge
and encouraging active discussion. Users, including patients, patient
advocates, clinicians and researchers, can participate, along with
community editors, in various stages of interpreting the clinical
significance of cancer variants using standards and guidelines
developed by community experts \cite{Li2017-aw,Griffith2016-sy}.

The Drug Gene Interaction Database (DGIdb) is an open source and open
access platform for gene and drug annotation for known interaction and
potential druggability. Users can can cross-reference genes of
interest and drugs against up to 15 sources and in functionally
classified gene categories
\cite{Wagner2016-fs,Griffith2013-uv}. Cancer Genome Interpreter (CGI)
identifies mutational events that are biomarkers of drug response or
interact with known chemical compounds
\cite{Tamborero2017-ay}. PharmGKB is a pharmacogenomic resource for
building clinical implementation and interpretation based on
annotating, integrating and aggregating knowledge extracted from
research-level publications. It provides scored clinical annotation,
prescription annotation (drug dosing, prescribing information), as
well as pharmacokinetics/pharmacodynamics (PK/PD) annotation, with
primary literature reference.

OncoKb contains information on the clinical implication of specific
genetic alterations in cancer.  Each variant is annotation from
multiple sources and scored using Levels of Evidence ranging from
Level 1, which includes FDA approved biomarker predictive of response
to an FDA-approved drug, to Level 2, which includes variants for which
an FDA-approved or standard of care treatment is available, Level 3
and Level 4 contain variants with investigational and hypothetical
therapeutic implications, respectively. A similarly structured scoring
system is available for indicating therapeutic implications for
variants associated with resistance \cite{Chakravarty2017-gx}. Cancer
Driver Log (CanDL), an expert-curated database for potential driver
mutations in cancer, employs a similar four-level scoring system based
on FDA approval, clinical, pre-clinical and experimental functional
evidence \cite{Damodaran2015-so}.

MyCancerGenome (MCG) is a knowledge resource highlighting the
implication of tumor mutation on cancer care. It allows users to
access its content via a mobile app and provide patient-focused
information. Patients can access a database entitled DNA-mutation
Inventory to Refine and Enhance Cancer Treatment (DIRECT) for
Epidermal Growth Factor Receptor (EGFR) mutation for non-small cell
lung cancer (NSCLC). Personalized Cancer Therapy (PCT) at the MD
Anderson Cancer Center is a resource for clinical response associated
with cancer variants and aims to facilitate patient involvement in
biomarker-related clinical trials. Drug effectiveness is associated
with a specific biomarker and scored based on prospective clinical
study as well as Food and Drug Administration (FDA) approval.

\section{Tools for manipulating variant datasets}

 
% vt
% http://genome.sph.umich.edu/wiki/Vt
% \cite{Tan2015-dv}
% bcftools
% http://www.htslib.org/download/
% \cite{Li2009-du}
% ANNOVAR
% http://annovar.openbioinformatics.org/en/latest/
% \cite{Wang2010-bt}
% SnpEff
% http://snpeff.sourceforge.net/
% \cite{Cingolani2012-pt}
% Oncotator
% https://portals.broadinstitute.org/oncotator/
% \cite{Ramos2015-vn}
% vcfanno
% https://github.com/brentp/vcfanno
% \cite{Pedersen2016-pu}


Processing sequence data with the goal of determining variants
(somatic or germline) often ends with a file in Variant Call Format
(VCF format), a loose, self-describing data standard describing
variants along a genome, associated statistical and numeric metrics
for each variant, and information integrated from data resources such
as those described in the preceding sections \cite{Danecek2011-du}. An
ecosystem of tools, listed in Table 4, has been developed for basic
transformations, manipulations, merge operations, and for adding
transcript, protein, and higher-level functional annotations to
variants in a VCF file. The vt and bcftools software suites perform
operations such as slicing by genomic coordinate, data compression,
and, importantly variant normalization, rendering variants more
readily comparable across resources. Annovar
\cite{Yang2015-bg,Wang2010-bt} and the SnpEff suite
\cite{Cingolani2012-pt} add annotations relative to gene annotations,
including information about transcript and protein-coding
changes. Recently, several software developers of variant annotation
tools have developed a standard for reporting gene-centric annotations
that has simplified post-processing of variants after
annotation. Finally, tools such as Vcfanno \cite{Pedersen2016-pu} have
been developed that can flexibly add fields to variants in a VCF file
based on relatively sophisticated logic and data transformations,
reducing the number of tools required to bring a new data resource
into the annotation pipeline.

\section{Pragmatic details}

Despite advanced toolsets for manipulating variant files and
increasing adoption available standard formats, practical pitfalls and
challenges remain to the basic manipulation of variant datasets. Some
data resources are available in multiple formats and not all formats
contain identical information. Matching variants between resources and
observed variants can be challenging, as some variants can be
represented validly in multiple forms. Ideally, variants are cataloged
with clarity with respect to a reference genome and, whenever
possible, using HGVS nomenclature \cite{Den_Dunnen2016-gw}. In spite
of increasing awareness and uptake of HGVS standard nomenclature, the
critical step of matching variants across tools and databases in
assessing clinical significance is still hampered by inconsistencies
across tools and databases \cite{Yen2017-lq}. Particularly when
handling clinical samples, an information system that provides results
from multiple resources when assessing novel variants, incorporates in
silico controls when adding or updating data resources (to avoid
introducing errors), and adheres to HGVS nomenclature wherever
possible in data processing pipelines can increase the likelihood of
discovering potentially relevant variants.

\section{Discussion}

Robust sequencing technologies and increasingly reliable
bioinformatics pipelines, combined with parallel development of
therapeutics and diagnostics has bolstered the field of precision
genomic oncology. However, the sheer number of resources available
that can inform the interpretation of small variants is staggering,
except for the very few variants with well-established clinical
relevance or an associated targeted therapy. This review has
highlighted a number of important data resources individually. For
other variants, data integration remains a significant hurdle to the
rapid turnaround required to apply HTS in a clinical context. Expert
panel review (the molecular tumor board) has been effective for some
groups \cite{Knepper2017-no,Beltran2015-pz,Sohal2015-bi} while other
groups have adopted a protocol-based approach
\cite{Ghazani2017-oo}. Even when molecularly targetable lesions are
identified, barriers to delivering therapy have been observed,
limiting the impact of precision genomic oncology in some settings
\cite{Bryce2017-ht}.

Cancer arises in an individual after a typically complex and
incompletely understood set of oncogenic events that are increasingly
observable at the molecular level. Progress in cancer prevention,
early detection, diagnosis, prognosis, and treatment is increasingly
driven by insight gained through the analysis and interpretation of
large genomic, proteomic, and pharmacological knowledge
bases. Reductionist approaches to cancer biology can achieve only
limited success in understanding cancer biology and improving
therapy. Cancer is a disease associated with disruption of normal
cellular circuitry and processes that leads to abnormal or
uncontrolled proliferative growth, characterized by a complex spectrum
of biochemical alterations that affects biological processes at
multiple scales from the molecular activity and cellular homeostasis
to intercellular and inter-tissue signaling. The cancer research
community has made great strides in measuring the oncogenic events
that lead to the development of cancer and therapy
resistance. However, the challenges brought by complexity inherent in
protein networks, intercellular signaling, cellular heterogeneity, and
the dynamic nature of cancer demand multiscale systems and modeling
approaches that address the interrelatedness of the biological
processes underlying cancer.

\section{Conflict of Interest}

The authors declare that the research was conducted in the absence of
any commercial or financial relationships that could be construed as a
potential conflict of interest.

\section{Acknowledgments}
This work was supported by the National Cancer Institute Center for
Biomedical Informatics and Information Technology and the National
Cancer Institute Center for Cancer Research in the Intramural Research
Program at the National Institutes of Health. Sections of this
manuscript have been adapted from the PhD thesis
\cite{Addepalli2014-oa} of D.A.

\printbibliography

\end{document}
