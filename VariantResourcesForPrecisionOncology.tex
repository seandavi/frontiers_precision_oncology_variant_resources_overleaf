\documentclass{article}

\usepackage[letterpaper,margin=1.25in]{geometry}
\usepackage{tabularx}
\usepackage{array}
\usepackage{booktabs}
\usepackage{fancyhdr}
\usepackage[textsize=tiny]{todonotes}
\usepackage{hyperref}
\usepackage[style=numeric,
sorting=none,
citestyle=authoryear,
backend=biber]{biblatex}
\usepackage{authblk}
\usepackage{tablefootnote}


\title{Resources for Interpreting Variants in Precision Genomic
Oncology Applications}

\addbibresource{references.bib}


\author[2,3]{Hsinyi Tsang}
\author[2,3]{Durga Addepalli}
\author[1]{Sean R. Davis\thanks{Address correspondence to: seandavi@gmail.com}}
\affil[1]{Center for Cancer Research, 
  National Cancer Institute, 
  National Institutes of Health, 
  Bethesda, MD, USA}
\affil[2]{Center for Biomedical Informatics and Information Technology, 
  National Cancer Institute, 
  National Institutes of Health, 
  Gaithersburg, MD, USA}
\affil[3]{Attain, LLC, 
  McClean, VA, USA} 

\renewcommand\Authands{ and }

\begin{document}

\maketitle


\begin{abstract} Precision genomic oncology--applying high throughput
sequencing (HTS) at the point-of-care to inform clinical decisions--is
a developing precision medicine paradigm that is seeing increasing
adoption. Simultaneously, new developments in targeted agents and
immunotherapy, when informed by rich genomic characterization, offer
potential benefit to a growing subset of patients. Multiple previous
studies have commented on methods for identifying both germline and
somatic variants. However, interpreting individual variants remains a
significant challenge, relying in large part on the integration of
observed variants with biological knowledge.  A number of data and
software resources have been developed to assist in interpreting
observed variants, determining their potential clinical actionability,
and augmenting them with ancillary information that can inform
clinical decisions and even generate new hypotheses for exploration in
the laboratory. Here, we review available variant catalogs, variant
and functional annotation software and tools, and databases of
clinically actionable variants that can be used in an ad hoc approach
with research samples or incorporated into a data platform for
interpreting and formally reporting clinical results.
\end{abstract}


\section{Introduction}
Genomic technologies and approaches have transformed cancer research
and have led to the production of large-scale cancer genomics
compendia
\parencite{noauthor_undated-vx,Cancer_Genome_Atlas_Research_Network2013-gt}. The
resulting molecular characterization and categorization of individual
samples from such compendia has driven development of molecular
subtypes cancers as well as enhanced understanding of the molecular
etiologies of carcinogenesis
\parencite{Cancer_Genome_Atlas_Network2012-nz,Cancer_Genome_Atlas_Research_Network2015-gd,noauthor_2008-wg}. The
development of novel and effective targeted therapies has proceeded in
parallel with and been accelerated by deeper, faster, and broader
genomic characterization \parencite{Blumenthal2016-fb}, enabling early
application of molecular characterization at the point of care to
inform clinical decision-making
\parencite{Flaherty2012-dq,Shaw2013-wl,Maemondo2010-dj,Druker2006-qk} and
to address resistance to primary therapy \parencite{Ai2014-nf}. Genomic
characterization also has applications in immune approaches to
cancer. For example, chimeric antigen receptor T-cell (CARt) therapy
have shown great success in diseases with well-characterized antigens
that are relatively tumor-specific \parencite{Grupp2013-nd} as identified
by genomic profiling. Variously referred to as precision oncology
\parencite{Sohal2015-bi}, genomics-driven oncology \parencite{Garraway2013-zo},
genomic oncology, and even simply as precision medicine, the paradigm
of applying high-throughput genomic approaches to patient samples is
rapidly changing the landscape of oncology care and clinical oncology
research.

Conventional approaches to clinical trials design may be inadequate
due to molecular heterogeneity of tumors derived from a single primary
tissue \parencite{Simon2016-ik}, leading to the adoption of basket,
umbrella, and hybrid trials designs. A number of studies are ongoing
to determine feasibility and potential impact of precision genomic
oncology at the point-of-care
\parencite{Cheng2015-wh,noauthor_undated-da,Lopez-Chavez2015-cg}. In
addition to studies focused on identifying targetable mutations,
immune-based therapeutic approaches are also being informed by HTS
applied to patient samples
\parencite{Bethune2017-ns,Chalmers2017-ya,Faltas2016-yz}.

One of the most recent developments in the field of precision oncology
is the approval of Pembrolizumab (Keytruda), a anti-PD-1 antibody that
functions as a checkpoint inhibitor, by the US Food and Drug
Administration for treatment of solid tumors that show genetic
evidence of mismatch repair and therefore carry very high mutational
burdens \parencite{Le2017-vc}. Pembrolizumab was previously approved for use in
melanoma, but the most recent approval is the first that is targeting
allows a drug to be used in a non-tissue-specific context in patients
showing a specific genomic marker in any solid tumor
\parencite{Garber2017-sk}. 

As with any clinical testing modality, whether in a research setting
or at the point-of-care, a clear understanding of the goals of
applying the test is necessary when first designing the test and its
validation. However, the flexibility and number of potential data
items that arise from even a limited application of HTS has lead the
US Food and Drug Administration (FDA) has begun to define its
regulatory role \parencite{Fda2015-kv} and, critically, how existing
knowledge bases can be applied in real time to address findings from
clinical HTS testing \parencite{Fda2016-kx}.

This review aims to provide an organized set of biological knowledge
bases with relevance to the interpretation of small variants, defined
as single nucleotide variants or short (on the order of 20 base pairs
or fewer) insertions and deletions. The catalogs of observed variants
section lists large-scale catalogs of variants, useful for filtering
known common polymorphisms and identifying previously-identified
cancer variants. When a variant observed in a clinical sample has not
been seen but appears to affect the protein coding sequence, the
functional annotation resources section presents a sampling of some of
the most common software and databases for predicting the impact on
protein function. Finally, we catalog several data products and
knowledgebases have been developed to provide decision support (with
strong disclaimers and caveats) directly linking observed variants to
clinical intervention in point-of-care HTS applications. Integrating
the various data sources described in this review with variants
observed in individual patients can be accomplished with combinations
of software tools for the manipulation of variant datasets.

\subsection{Catalogs of observed germline and somatic variants}

Databases of observed variation in normal populations,
diseased individuals, and cancer compendia form the map onto which
observed variants in patients are projected. Because of the vast
quantities of genomic data and, specifically, DNA variants, there is a
tension between providing rich, highly-curated information about
individual variants and producing the largest possible catalog of
variants with manageable levels of curation. This section reviews some
of the available catalogs (Table \ref{table:1}) of genomic variation observed
in the germline as well as those that appear in tumors as somatic
mutations.  Note that many of the databases mentioned below overlap in
data sources (some nearly completely), but they may differ in the
amount and depth of curation, additional metadata added to each
variant, speed of updates, and methods or formats for access.

\subsection{Germline}

Comprehensive catalogs of germline variants inform decisions
about the frequency of variants as seen in the general population as
well as to identify variants that are annotated as
cancer-associated. In the context of tumor sequencing, common variants
are unlikely to be genomic drivers of carcinogenesis and are often
filtered from a report of potential somatic variants. This filtering
process is particularly important when tumor sequencing is not
accompanied by matched normal sequencing. Additional germline
databases that catalog disease-associated variants can be useful to
begin to address familial risk and potentially pharmacogenomic loci
\parencite{Wheeler2013-dn,Relling2015-ie}.

Perhaps the oldest of the variant catalogs, dbSNP contains 325,658,303
individual variant records (build 150, accessed May 30, 2017) and is
available in multiple formats, searchable, and linked to records in
literature and other data resources and databases. While the vast
majority of variants in dbSNP have been observed in individuals
without cancer, somatic variants are included and annotated in the
database. Because dbSNP is driven by community submission of variants,
levels of evidence vary among individual variants. The genome
Aggregation Database, or gnomAD, \parencite{Lek2016-bb,noauthor_undated-of}
contains information from 123,136 exomes and 15,496 whole-genomes from
unrelated individuals sequenced as part of various disease-specific
and population genetic studies (accessed May 30, 2017). These data
were collected by numerous collaborations, underwent standard
processing, and unified quality control and results area accessible as
a searchable online database and as a downloadable VCF-format text
file. ClinVar \parencite{Landrum2016-ul}, maintained by the NIH National
Center for Biotechnology Information (NCBI), is a freely available
archive for interpretations of clinical significance of variants for
reported conditions. Entries in ClinVar are taken directly from
submitters and represent the relationship between variants and
clinical significance. When multiple submissions concerning a single
variant are available, ClinVar supplies high-level summaries of
agreement or disagreement across submitters. Importantly, though,
clinical significance in ClinVar is reported as supplied by the
submitter. The Personalized Genome Project \parencite{Church2005-lr}
provides a limited number of fully open-access genome sequencing
results provided by volunteers with trait surveys and even some
microbiome surveys of participants. A catalog of germline variants
derived from 69 genomes sequenced using the Complete Genomics
sequencing platform \parencite{Drmanac2010-od} may be useful for groups who
have data generated from the same platform, particularly for
identifying sequencing-platform-specific false positive results.

\subsection{Somatic}

Whereas databases of germline variants are useful to filter
out variants unlikely to be directly involved in carcinogenesis,
databases of somatic variants are useful to identify variants and
their frequencies as observed in tumors. In some cases, identified
variants may be associated with specific tumor types, offering
mechanistic clues, particularly in the rare cancer setting where
biological understanding may be limited.

Several catalogs of somatic variants have, at their core, variants
derived from The Cancer Genome Atlas (TCGA). These databases vary in
the pipelines used to define the variants, the level of annotation
associated with individual variants, the proportion of TCGA included,
and methods for accessing or querying. Recently, National Cancer
Institute (NCI) has established the Genomic Data Commons (GDC) to
harmonize clinical information and genomic results across enterprise
cancer datasets \parencite{Grossman2016-sk}, particularly those funded by
NCI, such as TCGA. In addition to the adult tumors profiled as part of
the TCGA, the NCI GDC also contains data from several pediatric tumors
profiled as part of the Therapeutically Applicable Research To
Generate Effective Treatments (TARGET) project
\parencite{noauthor_undated-ax}. Cancer cell line data from the Cancer Cell
Line Encyclopedia (CCLE) are also included \parencite{Barretina2012-yz} in
the GDC data collection. The GDC is a modern data platform that
provides multiple access methods including a programmatic application
programming interface (API), data file download, and web browser based
text and graphical queries and visualization. The International Cancer
Genome Consortium (ICGC) is a large, international collaboration with
a collection of 76 studies (including TCGA studies) encompassing 21
tissue primary sites. Like the NCI GDC, the ICGC data portal provides
modern data platform approaches to data access, visualization, and
query \parencite{Zhang2011-bl}. The Catalog of Somatic Mutations in Cancer
(COSMIC) database is perhaps the largest and best-known cancer variant
database. It presents a unified dataset consisting of curated cancer
variants for specific genes as well as genomic screens from projects
such as TCGA. Several other cancer variant data resources are listed
in Table \ref{table:1}.

\section{Functional Annotation Resources}

When faced with variants with little or no literature or database
support, differentiating those that variants that are likely to be
deleterious, perhaps contributing to carcinogenesis, versus those that
likely are tolerated by the cell is a critical task, particularly in
the setting of clinical precision genomic oncology. Note that
determing that a variant is deleterious is not likely to result in a
change in diagnosis, prognosis, or therapy. However, prioritizing
variants for further study, research interest, and for discussion in
forums such as a molecular tumor board is a valuable and necessary
aspect of applying genomic technologies in the clinical arena.

A number of algorithms and methods have been developed to predict the
effect of observed variants on protein structure and function as well
as the potential for clinical impact. These prediction methods utilize
features of the variant and its context such as sequence identity,
sequence conservation, evolutionary relationship, protein primary and
secondary structure, entropy based protein stability and approaches
such as clustering based on sequence alignments and machine
learning. Some of them are specific to the type of variant or
mutation, some to a disease type, and some more general. Therefore,
applying these functional annotational tools and interpreting the
results in a clinical or research setting may require significant
human curation before being recognized as clinically actionable. Here
we present a review of a representative set of approaches for
predicting pathogenicity of different variants. For a comprehensive
list of prediction tools and their details see Table
\ref{table:2}. For more detailed scientific and technical explanations
of these methods, we refer the reader to a comprehensive review
\parencite{Addepalli2014-oa}.

\subsection{SIFT}

Sorting Intolerant From Tolerant, or SIFT, predicts functional impacts
of amino acid substitutions \parencite{Ng2003-vp} is one of the earliest
variant effect prediction tools and represents the class of prediction
algorithms that utilizes protein conservation. It has since been
updated and an online version of the tool is available
\parencite{Kumar2009-gd}. SIFT uses sequence homology, as measured by
protein-level conservation, to classify variants based as tolerated or
deleterious based on the associated protein coding changes. SIFT has
served as a benchmark against which other methods are compared because
of its relative simplicity. SIFT considers the type of amino acid
change induced by a genomic variant and the position at which the
change/mutation occurs. SIFT relies on the presence of sequences from
which conservation can be determined; variants for which such
databases are limited will potentially lack robust predictions.

\subsection{PolyPhen-2}

Polymorphism Phenotyping v2, or PolyPhen2, predicts the effecting of
coding nonsynonymous SNPs on protein structure and function and
annotates them \parencite{Adzhubei2013-nj}. This algorithm uses a naive
Bayes approach to combine information across a panel of 3D structural,
sequence-based, and conservation-based features. Trained on two
datasets, HumDiv and HumVar, and associated non-deleterious controls,
the PolyPhen2 algorithm represents a class of multivariate prediction
algorithms that employ machine learning and multiple features of
variant impact. 

\subsection{Mutation Assessor}

Mutation Assessor is an algorithm and tool that, like SIFT, uses a
conservation-based approach. However, Mutation Assessor also
incorporates evolutionary information in an attempt to account for
shifts in function between subfamilies of proteins \parencite{Reva2011-en},
potentially extending the functional annotation of variants to
``switch of function'' as well as loss or gain of function. By
quantifying the impact to conserved residues both globally and within
subfamilies (residues that distinguish subfamilies from each other are
thought to be less tolerant to change), Mutation Assessor defines a
functional impact score to predict which variants are likely to be
deleterious.

\subsection{CONDEL}

The CONsensus DELeteriousness, or CONDEL score, is an integrated
prediction method for missense mutations that is relatively easy to
extend with additional prediction resources
\parencite{Gonzalez-Perez2011-gc}. Originally implemented as a weighted
average of the normalized scores from the output of two computational
tools, Mutation Assessor and FATHMM, CONDEL can be extended or adapted
to data at hand and represents an ``aggregator'' approach to variant
effect prediction.  Condel scores can be derived for a limited set of
specified mutations via an online web application. The Ensembl
database provides a variation of position-specific CONDEL predictions
that combine SIFT and Polyphen-2 for every possible amino acid
substitution in all human proteins.

\subsection{CHASM}

Cancer-specific High-throughput Annotation of Somatic Mutations, or
CHASM, is a computational method that identifies and prioritizes the
missense mutations likely to enhance tumor cell proliferation
\parencite{Carter2009-ci}. CHASM uses machine learning to classify putative
``driver'' cancer mutations as distinct from ``passenger''
mutations. Training the CHASM model employed in-silico simulation to
generate realistic ``passenger'' mutations, specifically modeled to
represent variant context and genes that are observed in cancer
settings. Multiple features of the variants including their DNA and protein
contexts were then used to build a machine learning approach that
attempted to maximize the specificity of separating driver mutations
from passenger mutations. CHASM represents a relatively specific
algorithm focused not on ``deleteriousness'' but, rather, on the
likelihood that an observed variant is a cancer ``driver''.


\subsection{dbNSFP}

Recognizing that applying all of the effect prediction tools available
is potentially challenging, \cite{Liu2016-iv} developed a database
that aggregates predictions for \textit{all} possible SNVs associated
with coding changes (in Gencode gene models). With more than ten
different prediction algorithms and extensive additional annotation,
this database can be a useful one-stop-shop for adding annotations to
variant datasets. The snpEff suite (described below) can be used in
conjunction with dbNSFP to efficiently annotate SNPs with the
potential to effect coding genes.

\section{Clinical Actionability}



The ultimate goal for many of the above-mentioned resources is to
develop an individualized approach to the diagnosis, prevention and
treatment of cancer, or precision oncology. However, despite recent
advances in HTS, determining the clinical relevance of experimentally
observed cancer variants remains a challenge in the application of HTS
in clinical practice. Difficulties in differentiating driver and
passenger mutations, lack of standards and guidelines in reporting and
interpretation of genomic variants, lack of clinical evidence in
associating genomic variants to clinical outcome, lack of resources to
disseminate clinical knowledge to the cancer community and the precise
definition of actionability have been reported to contribute to the
bottleneck
\parencite{Li2017-aw,Prawira2017-gv,Uzilov2016-ct,Hedley_Carr2016-ul}. Comprehensive
resources linking experimentally determined cancer variants and
clinical actionability have been developed to address some of these
challenges and address various aspects of translating research results
into clinical valuable information to support clinical decisions in
precision oncology (See Table \ref{table:3}). In recognition of the fact that
central curation of information regarding actionability is extremely
challenging, several of the resources below use crowdsourcing as a
means of gathering updates and enhancing curation efforts. In addition
to a web interface, some tools provide additional access via API,
mobile app, and/or social media tagging to facilitate dissemination of
information and enhance accessibility. While some of these tools share
similar functions, in the section below, we highlight distinct
features and capabilities for a representative set of resources that
might be used as a ``starter'' set for clinical annotation of
variants.

The myvariant.info database is one of the newest and attempts to
provide a ``one-stop-shop'' for variants. It is included in this
section because it has recently incorporated the CIViC and Cancer
Genome Interpreter databases. In addition, it provides annotations for
SNVs from multiple other data sources (a growing list, so see the site
for updates) and aggregates functional annotations for variants
present in its database, making it a good all-around tool for cancer
variant annotation. It is available as a performant web API only at
this time. 

Clinical Interpretation of Variants in Cancer (CIViC) is an open
access and open source platform for community-driven curation and
interpretation of cancer variants. It is based on a crowdsourcing
model where individuals in the community can contribute to produce a
centralized knowledge base with the goal of disseminating knowledge
and encouraging active discussion. Users, including patients, patient
advocates, clinicians and researchers, can participate, along with
community editors, in various stages of interpreting the clinical
significance of cancer variants using standards and guidelines
developed by community experts \parencite{Li2017-aw,Griffith2016-sy}.

The Drug Gene Interaction Database (DGIdb) is an open source and open
access platform for gene and drug annotation for known interaction and
potential druggability. Users can can cross-reference genes of
interest and drugs against up to 15 sources and in functionally
classified gene categories
\parencite{Wagner2016-fs,Griffith2013-uv}. Cancer Genome Interpreter (CGI)
identifies mutational events that are biomarkers of drug response or
interact with known chemical compounds
\parencite{Tamborero2017-ay}. PharmGKB is a pharmacogenomic resource for
building clinical implementation and interpretation based on
annotating, integrating and aggregating knowledge extracted from
research-level publications. It provides scored clinical annotation,
prescription annotation (drug dosing, prescribing information), as
well as pharmacokinetics/pharmacodynamics (PK/PD) annotation, with
primary literature reference.

OncoKb contains information on the clinical implication of specific
genetic alterations in cancer.  Each variant is annotation from
multiple sources and scored using Levels of Evidence ranging from
Level 1, which includes FDA approved biomarker predictive of response
to an FDA-approved drug, to Level 2, which includes variants for which
an FDA-approved or standard of care treatment is available, Level 3
and Level 4 contain variants with investigational and hypothetical
therapeutic implications, respectively. A similarly structured scoring
system is available for indicating therapeutic implications for
variants associated with resistance \parencite{Chakravarty2017-gx}. Cancer
Driver Log (CanDL), an expert-curated database for potential driver
mutations in cancer, employs a similar four-level scoring system based
on FDA approval, clinical, pre-clinical and experimental functional
evidence \parencite{Damodaran2015-so}.

MyCancerGenome (MCG) is a knowledge resource highlighting the
implication of tumor mutation on cancer care. It allows users to
access its content via a mobile app and provide patient-focused
information. Patients can access a database entitled DNA-mutation
Inventory to Refine and Enhance Cancer Treatment (DIRECT) for
Epidermal Growth Factor Receptor (EGFR) mutation for non-small cell
lung cancer (NSCLC). Personalized Cancer Therapy (PCT) at the MD
Anderson Cancer Center is a resource for clinical response associated
with cancer variants and aims to facilitate patient involvement in
biomarker-related clinical trials. Drug effectiveness is associated
with a specific biomarker and scored based on prospective clinical
study as well as Food and Drug Administration (FDA) approval.

\section{Tools for manipulating variant datasets}

 

Processing sequence data with the goal of determining variants
(somatic or germline) often ends with a file in Variant Call Format
(VCF format), a loose, self-describing data standard describing
variants along a genome, associated statistical and numeric metrics
for each variant, and information integrated from data resources such
as those described in the preceding sections \parencite{Danecek2011-du}. An
ecosystem of tools, listed in Table \ref{table:4}, has been developed for basic
transformations, manipulations, merge operations, and for adding
transcript, protein, and higher-level functional annotations to
variants in a VCF file. The vt and bcftools software suites perform
operations such as slicing by genomic coordinate, data compression,
and, importantly variant normalization, rendering variants more
readily comparable across resources. Annovar
\parencite{Yang2015-bg,Wang2010-bt} and the SnpEff suite
\parencite{Cingolani2012-pt} add annotations relative to gene annotations,
including information about transcript and protein-coding
changes. Recently, several software developers of variant annotation
tools have developed a standard for reporting gene-centric annotations
that has simplified post-processing of variants after
annotation. Finally, tools such as Vcfanno \parencite{Pedersen2016-pu} have
been developed that can flexibly add fields to variants in a VCF file
based on relatively sophisticated logic and data transformations,
reducing the number of tools required to bring a new data resource
into the annotation pipeline.


\section{Discussion}

\subsection{Pragmatic details}

Despite advanced toolsets for manipulating variant files and
increasing adoption available standard formats, practical pitfalls and
challenges remain to the basic manipulation of variant datasets. Some
data resources are available in multiple formats and not all formats
contain identical information. Matching variants between resources and
observed variants can be challenging, as some variants can be
represented validly in multiple forms. Ideally, variants are cataloged
with clarity with respect to a reference genome and, whenever
possible, using HGVS nomenclature \parencite{Den_Dunnen2016-gw}. In spite
of increasing awareness and uptake of HGVS standard nomenclature, the
critical step of matching variants across tools and databases in
assessing clinical significance is still hampered by inconsistencies
across tools and databases \parencite{Yen2017-lq}. Particularly when
handling clinical samples, an information system that provides results
from multiple resources when assessing novel variants, incorporates in
silico controls when adding or updating data resources (to avoid
introducing errors), and adheres to HGVS nomenclature wherever
possible in data processing pipelines can increase the likelihood of
discovering potentially relevant variants.

This review is meant to be comprehensive, so the reader might wonder
``Where do we start?''. It is difficult to make hard-and-fast
recommendations about what resources, tools, and databases are ``the
best'' given the lack of gold-standard datasets on which to base such
evalutations. Furthermore, the context for sequencing (clinical or
not, targeted mutations, trial setting, or novel variant and biomarker
discovery) will drive annotation pipeline development. Not all data resources need
to be added simultaneously if developing a pipeline for annotating
cancer variants for precision oncology applications. In a clinical
setting, targeting the reporting workflow and working with clinicians
to understand the most relevant annotations is the most efficient
approach to determining relevant resources for annotation. In the more
general sequencing applications, starting with some key resource
(dbSNP, COSMIC, ClinVar, dbNSFP, for example) and
extending with additional resources as they become available or as
time and interest permits is a reasonable approach. Developing a modular
informatics pipeline, perhaps using a computational workflow framework
(\url{https://github.com/pditommaso/awesome-pipeline}) that can be easily
extended and re-run on previously annotated data is helpful to keep
pace with the rapidly changing and growing collection of annotation
resources. Newer aggregation resources such as myvariant.info offer a
wholistic solution (annotation, catalog, and clinical actionability),
but with some risk of ``lossiness'' with respect to the primary
resources contained within.

Finally, given the rapid pace of new development in this space, we
have established a crowd-sourced list of cancer variant resources for
precision medicine available at
\url{https://github.com/seandavi/awesome-cancer-variant-databases}. 

\subsection{Conclusion}

Robust sequencing technologies and increasingly reliable
bioinformatics pipelines, combined with parallel development of
therapeutics and diagnostics has bolstered the field of precision
genomic oncology. However, the sheer number of resources available
that can inform the interpretation of small variants is staggering,
except for the very few variants with well-established clinical
relevance or an associated targeted therapy. This review has
highlighted a number of important data resources individually. For
other variants, data integration remains a significant hurdle to the
rapid turnaround required to apply HTS in a clinical context. Expert
panel review (the molecular tumor board) has been effective for some
groups \parencite{Knepper2017-no,Beltran2015-pz,Sohal2015-bi} while other
groups have adopted a protocol-based approach
\parencite{Ghazani2017-oo}. Even when molecularly targetable lesions are
identified, barriers to delivering therapy have been observed,
limiting the impact of precision genomic oncology in some settings
\parencite{Bryce2017-ht}. Not covered in this
review is the increasing utility of HTS in the burgeoning field of
immunotherapy, where early efforts to predict response based on HTS
results have been promising
\parencite{Wang2017-yd,Yarchoan2017-vl,Bethune2017-ns}. 


Some interesting trends are evident in the databases and resources
presented in this review that highlight the overarching trends in delivering
precision medicine. First is the sheer volume and rapid growth of
numbers of
observations to learn about the spectrum of variation
cancer and normal genomes. Projects like GnomAD, COSMIC, and other
data sharing efforts enhance precision by cataloging rare variants as
well as precise estimates of the frequencies of common
variants. Second is the use of
crowd-sourcing to produce rich clinical annotation (eg., CiVIC) in
response to the need for intensive human interaction to interpret the clinical impact of a variant or its relationship to
potential medical intervention. On the other hand, with volumes of data ever-increasing, machine learning
techniques drive many of the most commonly-used approaches for
assigning scores for impact of observed variants. As well-annotated
datasets and variant catalogs grow, application of machine learning
will become both more common and more powerful.

While significant progress has been made in applying technology to
precision oncology, cancer arises in an individual after a typically complex and
incompletely understood set of oncogenic events that are increasingly
observable at the molecular level. Progress in cancer prevention,
early detection, diagnosis, prognosis, and treatment is increasingly
driven by insight gained through the analysis and interpretation of
large genomic, proteomic, and pharmacological knowledge
bases. Reductionist approaches to cancer biology can achieve only
limited success in understanding cancer biology and improving
therapy. Cancer is a disease associated with disruption of normal
cellular circuitry and processes that leads to abnormal or
uncontrolled proliferative growth, characterized by a complex spectrum
of biochemical alterations that affects biological processes at
multiple scales from the molecular activity and cellular homeostasis
to intercellular and inter-tissue signaling. The cancer research
community has made great strides in measuring the oncogenic events
that lead to the development of cancer and therapy
resistance. Because of the complexity inherent in
protein networks, intercellular signaling, cellular heterogeneity, and
the dynamic nature of cancer, future progress will require a more
wholistic approach to precision oncology including multiscale systems and modeling
approaches that address the interrelatedness of the biological
processes underlying cancer.

\section{Conflict of Interest}

This work was performed while KA and HT were employed by Attain, LLC,
in support of bioinformatics projects at the National Cancer
Institute. The authors declare that the work was conducted in the
absence of any commercial or financial relationships that constitute a
potential conflict of interest.

\section{Acknowledgments}
This work was supported by the National Cancer Institute Center for
Biomedical Informatics and Information Technology and the National
Cancer Institute Center for Cancer Research in the Intramural Research
Program at the National Institutes of Health. 

\printbibliography

\pagebreak

\begin{table}
  \footnotesize
\centering
\begin{tabular}{>{\raggedright}p{1.25in}p{1.25in}p{1.5in}p{1.25in}<{\raggedright}}
  \toprule
  Resource & Variant Type & URL & Citation \\
  \midrule
  dbSNP\textsuperscript{1} & Germline and somatic &
  \url{https://www.ncbi.nlm.nih.gov/projects/SNP/} &
  \parencite{Sherry2001-li} \\
  COSMIC\textsuperscript{1} & Somatic & \url{http://cancer.sanger.ac.uk/cosmic} &
  \parencite{Reva2011-en} \\
  ClinVar\textsuperscript{1} & Germline predisposition and somatic & \url{https://www.ncbi.nlm.nih.gov/clinvar/intro/} & \parencite{Landrum2016-ul} \\
  gnomAD\textsuperscript{2}  & Germline & \url{http://gnomad.broadinstitute.org/} & \parencite{Lek2016-bb} \\
  69 genomes from CGI\textsuperscript{3} & Germline & \url{http://www.completegenomics.com/public-data/69-genomes/} & \parencite{Drmanac2010-od} \\
  Personalized Genome Project & Germline & \url{http://www.personalgenomes.org/} & \parencite{Church2005-lr} \\
  NCI Genomic Data Commons & Germline and somatic & \url{https://portal.gdc.cancer.gov/} & \parencite{Grossman2016-sk} \\
  cBioPortal & Somatic & \url{http://www.cbioportal.org} & \parencite{Cerami2012-el,Gao2013-li} \\
  Intogen (Partial TCGA dataset) & Somatic & \url{https://www.intogen.org/search} & \parencite{Rubio-Perez2015-ek,Gonzalez-Perez2013-cl} \\
  Pediatric Cancer Genome Project  & Somatic & \url{http://explorepcgp.org} & \parencite{Downing2012-do} \\
  \bottomrule
\end{tabular}
\caption{Catalogs of germline and somatic variants. The most commonly
  used catalogs include dbSNP, COSMIC, ClinVar, and
  gnomAD. \textsuperscript{1}Primary
  resources useful for all studies. \textsuperscript{2}Particularly useful for
  exome sequencing projects. \textsuperscript{3}Useful if
  the Complete Genomics platform was used.} 
\label{table:1}
\end{table}


\begin{table}[p]
  \footnotesize
\centering
\begin{tabularx}{1\textwidth}{p{2.5cm}Xp{3.5cm}p{3.5cm}}
  \toprule
Resource & URL & Citation & Notes \\
  \midrule
  \multicolumn{4}{l}{Integrated predictive methods and aggregated databases}    \\
  \midrule
  dbNSFP\textsuperscript{1,2,3,4} & \url{https://sites.google.com/site/jpopgen/dbNSFP} &
  \parencite{Liu2016-iv} & Aggregated database of variant information
  \\
  myvariant.info\textsuperscript{1} & \url{http://myvariant.info/} &
  \parencite{Xin2016-yc} & Aggregated database of variant information \\
  \midrule
  \multicolumn{4}{l}{Functional effect prediction software and algorithms}    \\
  \midrule
  PolyPhen-2\textsuperscript{2} & \url{http://genetics.bwh.harvard.edu/pph2} & \parencite{Adzhubei2013-nj} & Bayesian classification \\
  SIFT\textsuperscript{2} & \url{http://sift.jcvi.org} & \parencite{Ng2003-vp} & Alignment scores \\
  MutationAssessor & \url{http://mutationassessor.org} & \parencite{Reva2011-en} & conservation, naive Bayes classifier \\
  MutationTaster & \url{http://www.mutationtaster.org} & \parencite{Schwarz2014-ep} &   \\
  PROVEAN & \url{http://provean.jcvi.org/index.php} & \parencite{Choi2012-tk} &   \\
  CADD\textsuperscript{2,3} & \url{http://cadd.gs.washington.edu} & \parencite{Kircher2014-im} &   \\
  GERP++\textsuperscript{3} & \url{http://mendel.stanford.edu/SidowLab/downloads/gerp/index.html} & \parencite{Davydov2010-ui} &   \\
  PhyloP and PhastCons & \url{http://compgen.cshl.edu/phast/index.php} & \parencite{Siepel2005-ke,Pollard2010-of} &   \\
  nsSNPAnalyzer & \url{http://snpanalyzer.uthsc.edu/} & \parencite{Bao2005-jn} & Random Forest \\
  SNPs\&GO & \url{http://snps-and-go.biocomp.unibo.it/snps-and-go/} & \parencite{Calabrese2009-mi} & SVM \\
  SNAP2 & \url{https://rostlab.org/services/snap2web/} & \parencite{Hecht2015-ti} & Neural Networks \\
  SNPs3D & \url{http://www.snps3d.org/} & \parencite{Yue2006-kj} & Structure and sequence analysis \\
  MutPred2 & \url{http://mutpred.mutdb.org/} & \parencite{Pejaver2017-lr} & Random Forest \\
  AUTO-MUTE & \url{http://binf2.gmu.edu/automute/} & \parencite{Masso2010-gf} & Topology and statistical contact potential\\ 
  Panther & \url{http://www.pantherdb.org/tools/csnpScoreForm.jsp} & \parencite{Thomas2003-rj} & Hidden Markov Model \\
  stSNP & \url{http://ilyinlab.org/StSNP/} & \parencite{Uzun2007-bk} &
  comparative modelling of protein structure \\
  Condel\textsuperscript{2} & \url{http://bg.upf.edu/fannsdb/} & \parencite{Gonzalez-Perez2011-gc} & a weighted average of multiple methods \\
  CoVEC & \url{https://sourceforge.net/projects/covec/files} &   &   \\
  CAROL\textsuperscript{2} & \url{http://www.sanger.ac.uk/science/tools/carol} &
  \parencite{Lopes2012-je} & combines PolyPhen-2 and SIFT \\
  \midrule
  \multicolumn{4}{l}{Cancer-specific prediction tools}    \\
  \midrule
  CHASM & \url{http://wiki.chasmsoftware.org/index.php/Main_Page} &
  \parencite{Carter2009-ci} & Random Forest \\
  CanDrA & \url{http://bioinformatics.mdanderson.org/main/CanDrA\#CanDrA} & \parencite{Mao2013-ie} & 96 structural, evolutionary and gene features  \\
   \bottomrule
\end{tabularx}
\caption{Tools, software, and databases for functional prediction and
  annotation of variant impact. \textsuperscript{1}Aggregated
  databases combine outputs of other databases and algorithms are,
  therefore, efficient resources to use in annotation
  pipelines. Adding these resources to observed variants is
  supported software in table \ref{table:4} including Ensembl VEP software (noted\textsuperscript{2} in
  this table), Annovar (noted\textsuperscript{3}), and snpEff
  (noted\textsuperscript{4}).
}
\label{table:2}
\end{table}



\begin{table}[p]
  \footnotesize
\centering
\begin{tabularx}{\textwidth}{p{3cm}XXp{1.5cm}p{1.5cm}}
  \hline
  Resource & URL & Citation & Crowd-sourcing used & Buik Access \\
  \hline
  myvariant.info\textsuperscript{*} & \url{http://myvariant.info/} &
  \parencite{Xin2016-yc} & Yes & API \\
  CIViC\textsuperscript{*} & \url{https://civic.genome.wustl.edu/home} &
  \parencite{Griffith2016-sy} & Yes & API, Download \\
  DGIdb\textsuperscript{*} & \url{http://dgidb.genome.wustl.edu/} &
  \parencite{Wagner2016-fs,Griffith2013-uv} & Yes & API, Download\\
  Cancer Genome Interpreter\textsuperscript{*} & \url{https://www.cancergenomeinterpreter.org/home} & \parencite{Tamborero2017-ay} & Yes & API \\
  OncoKb\textsuperscript{*} & \url{http://oncokb.org/} & \parencite{Chakravarty2017-gx} &
  & API \\
  Cancer Driver Log & \url{https://candl.osu.edu/} & \parencite{Damodaran2015-so} & Yes & Download\\
  Clinical Knowledge Base &
  \url{https://www.jax.org/clinical-genomics/clinical-offerings/ckb} &
  & & \\
  My Cancer Genome & \url{http://www.mycancergenome.org} &
  \parencite{Micheel2014-pz} & Yes & (licensed) API \\
  Personalized Cancer Therapy & \url{https://pct.mdanderson.org} & &
  Account required \\
  PharmGKB & \url{https://www.pharmgkb.org/} &
  \parencite{Hewett2002-yu} & Yes & Download \\
  Precision Medicine Knowledge Base (Beta) & \url{https://pmkb.weill.cornell.edu/} & \parencite{Huang2016-zx} & Yes & \\
  \hline
\end{tabularx}
\caption{In a clinical setting, these databases
are the most relevant, as they are maintained to provide clinically
actionable and curated content. While evalutation of each database by both
clinical and informatics team members, databases marked with ``*'' are
maintained, recently (or continuously) updated, and curated. The
myvariant.info database includes both CiVIC and Cancer Genome
Interpreter data. The last column in the table notes bulk access
approaches as these are relevant when including databases in an
annotation pipeline or automated report. }
\label{table:3}
\end{table}
 
 
\begin{table}[p]
  \footnotesize
\centering
\begin{tabularx}{\textwidth}{p{2.5cm}XXp{1.5cm}p{1.5cm}}
  \hline
  Software & URL & Citation \\
  \hline
  vt & \url{http://genome.sph.umich.edu/wiki/Vt} & \parencite{Tan2015-dv} \\
  bcftools & \url{http://www.htslib.org/download/} & \parencite{Li2009-du} \\
  ANNOVAR & \url{http://annovar.openbioinformatics.org/en/latest/} & \parencite{Wang2010-bt} \\
  Ensembl Variant Effect Predictor (VEP) & \url{http://www.ensembl.org/vep} & \parencite{McLaren2016-br} \\
  SnpEff & \url{http://snpeff.sourceforge.net/} & \parencite{Cingolani2012-pt} \\
  Oncotator & \url{https://portals.broadinstitute.org/oncotator/} &
  \parencite{Ramos2015-vn} \\
  vcfanno & \url{https://github.com/brentp/vcfanno} &
  \parencite{Pedersen2016-pu} \\
  \hline
\end{tabularx}
\caption{Software tools for manipulating and adding annotations to
  variant datasets. Variant calling produces a list of observed variants. The
  tools in this table are useful for adding biological interpretation
  and for annotating the variants with information from resources in
  tables \ref{table:1}, \ref{table:2}, and \ref{table:3}.} 
\label{table:4}
\end{table}




\end{document}
